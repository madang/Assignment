%% LyX 2.0.6 created this file.  For more info, see http://www.lyx.org/.
%% Do not edit unless you really know what you are doing.
\documentclass[english]{IEEEtran}
\usepackage[T1]{fontenc}
\usepackage[utf8]{luainputenc}

\makeatletter

%%%%%%%%%%%%%%%%%%%%%%%%%%%%%% LyX specific LaTeX commands.
%% Because html converters don't know tabularnewline
\providecommand{\tabularnewline}{\\}

\makeatother

\usepackage{babel}
\begin{document}

\title{A more physical way to make digitally reconstructed radiographs}


\IEEEaftertitletext{An appendix to the 2D-3D registration report}


\author{Hennadii Madan}
\maketitle
\begin{abstract}
We investigate whether a more physically correct way to calculate
intensities of digitally reconstructed radiographs will give any benefits
to registration.
\end{abstract}

\section{Introduction}

During preparation of the report for young researcher candidate assignment
in image processing laboratory, faculty of electrical engineering,
Ljubljana University I have noticed that using formula {*}INTEGRAL{*}
does not make physical sense. After some simple calculation it became
clear that a more physically correct way to obtain a digitally reconstructed
radiograph (DRR) would be to use {*}FORMULA E\textasciicircum{}-INTEGRAL{*}.
It is uclear wether this new way will improve registration, after
all it is just an intensity mapping. Due to time constraints this
idea was not tested during the assignment. Now I set to find this
out.


\section{Preparations}

A function to create DRRs {*}drr.m{*} has been augmented to accept
the sixth logical (MATLAB way to say 'boolean') paremeter which would
trigger exponentiation. Since the units of the coefficient of attenuation
are used in this CT scan are unknown I had to introduce a scaling
constant (godCoeff in the code). It would scale the resulting images
approximately to the range 10:255. A slightly modifie copy of OPTIM.m
optimization script named OptimAdj.m has been created. Optimizator
parameters were the same as in the original report. That is 'MaxIter'
= 200. 'TolX' = 1e-4,'TolFun' =1e-5 both for CC and MI.


\section{Results}


\subsection{Correlation coefficient}

Astonishingly (for me) after the very first iteration, i.e. after
a single run of fminunc from the strating point {[}0 0 0 0 0 0{]}
a CC of 0.7. Optimization log saved as ADJ\_OPT\_LOG.mat


\section{Discussion}

So CC registration indeed improved. It is reasonble since for example
two exponents corellate better than a line and an exponent. Not only
does it affect the magnitude of cerellation coefficient, but also
the resulting alignment upon optimization. I explain this to myself
the following way. When one changes the intensity mapping in an image
in effect he makes some features more prominent, the algorythm would
make these features match with features that are prominent in the
unaltered images. Since in general these are different we will obtain
a worse alignment then when we have perfectly matchin intensity mappings.


\section{Extras}

\begin{table}


\caption{CC optimization log}
\begin{tabular}{|c|c|c|c|c|c|c|c|}
\hline 
 &  &  &  &  &  &  & \tabularnewline
\hline 
\hline 
 &  &  &  &  &  &  & \tabularnewline
\hline 
 &  &  &  &  &  &  & \tabularnewline
\hline 
 &  &  &  &  &  &  & \tabularnewline
\hline 
 &  &  &  &  &  &  & \tabularnewline
\hline 
 &  &  &  &  &  &  & \tabularnewline
\hline 
 &  &  &  &  &  &  & \tabularnewline
\hline 
 &  &  &  &  &  &  & \tabularnewline
\hline 
 &  &  &  &  &  &  & \tabularnewline
\hline 
 &  &  &  &  &  &  & \tabularnewline
\hline 
 &  &  &  &  &  &  & \tabularnewline
\hline 
 &  &  &  &  &  &  & \tabularnewline
\hline 
 &  &  &  &  &  &  & \tabularnewline
\hline 
 &  &  &  &  &  &  & \tabularnewline
\hline 
 &  &  &  &  &  &  & \tabularnewline
\hline 
 &  &  &  &  &  &  & \tabularnewline
\hline 
 &  &  &  &  &  &  & \tabularnewline
\hline 
 &  &  &  &  &  &  & \tabularnewline
\hline 
 &  &  &  &  &  &  & \tabularnewline
\hline 
 &  &  &  &  &  &  & \tabularnewline
\hline 
 &  &  &  &  &  &  & \tabularnewline
\hline 
 &  &  &  &  &  &  & \tabularnewline
\hline 
 &  &  &  &  &  &  & \tabularnewline
\hline 
 &  &  &  &  &  &  & \tabularnewline
\hline 
\end{tabular}

\end{table}

\end{document}
